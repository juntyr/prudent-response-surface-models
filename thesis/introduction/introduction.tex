\chapter{Introduction}

Response Surface Models (RSMs) are fast, reduced complexity models that are fit to approximate the response of a complex higher-level model to changes in its inputs. They have been widely applied in areas ranging from experimentally determining the optimal conditions of a process \cite{response-surface-modelling-1951} such as pizza cheese \cite{pizza-optimisation-2018}, identifying the key factors that influence the temperature in subway tunnels \cite{subway-temperature-rsm-2019}, evaluating the energy production of geothermal reservoirs \cite{geothermal-rsm-2016}, and predicting the effect of new anaesthesia drugs on patients \cite{anesthesia-rsm-2015}. In this thesis, we focus on examples in atmospheric chemistry, where RSMs have already been applied to air quality models to make assessing the impacts of new policies more accessible to urban developers and decision makers (see, e.g. \cite{rsm-epa-2006, pf-rsm-2018, deep-rsm-2020}).

Machine Learning (ML) has rapidly increased in capability and popularity over the past decade \cite{ml-trends-2021, ml-applications-2021}. Research groups outside of data science are increasingly attempting to harness the runtime efficiency of ML and neural networks (NNs) in particular to speed up data analysis and inference \cite{srsm-2004}, integrate their highly complex process models into larger scale simulations such as global climate models \cite{hydrology-ml-2021, hybrid-model-2022}, or combine or replace them with ML-based emulators that aim to provide a live digital replica simulation of our world \cite{simulation-ml-2022, virtual-laboratories-2022}. Thus, machine learning models are going to become embedded in all areas of science and increasingly provide the results that the scientific literature and policymakers rely upon. However, machine learning methods are also known to easily reinforce discriminatory biases \cite{ml-bias-discrimination-2017, prediction-inequity-2019}, overconfidently make dangerous mistakes \cite{ood-baseline-2016}, and be confused by just a few well-placed pixels to mistake an image of a panda as a gibbon \cite{aversarial-attacks-2015}.

\newpar The primary goal of this thesis is to explore how response surface models can be built and evaluated such that their predictions can be trusted and safely used in scientific research. Toward this goal, we introduce a framework for constructing \textit{prudent} response surface models that can be integrated into scientific analyses with \textit{confidence} and \textit{certainty}. In particular, we design the Icarus RSM architecture that combines three existing methods to make predictions that are conditioned on a confidence level and consist of both the target value estimate and its uncertainty range.

This thesis puts particular focus on the area of out-of-distribution (OOD) detection, which encompasses methods used to detect model inputs that look very different from those used to fit the RSM. Detecting OOD inputs is crucial since an RSM generally cannot make a valid prediction or uncertainty estimate for them. We develop three simple and easy-to-visualise toy examples that can be used to evaluate different OOD detectors and highlight those that work in simple low-dimensional cases. We also propose several improvements for training an OOD detector and confidence scorer.

\newpar We take the atmospheric chemistry transport model SOSAA \cite{sosa-description-2011}, the model to \textbf{S}imulate \textbf{O}rganic vapours, \textbf{S}ulphuric \textbf{A}cid and \textbf{A}erosols, as an example of a slow and complex model in this project. Modelling atmospheric chemistry is vital to understanding the impacts that anthropogenic emissions and our changing environment have on human health and climate change. We explore several implementations for each of the three components of the Icarus RSM, provide recommendations for where they excel or fail, and combine them in a proof-of-concept implementation of a new SOSAA RSM. We use this practical implementation of the SOSAA RSM to demonstrate the promises and remaining limitations of implementing an Icarus-based RSM in practice. We also design a graphical user interface to configure the SOSAA model and experiment with the SOSAA RSM to make both more accessible for researchers in the future.

\newpar We have conducted a wide range of experiments and analyses for this thesis. All data, code, results, and visualisations produced during this project can be found at \href{https://github.com/juntyr/msc-tcm}{https://github.com/juntyr/msc-tcm}.

\newpage

\newpar The main contributions of this Master's Thesis are:
\begin{enumerate}
    \item We introduce the Icarus RSM architecture (\Cref{txt:icarus-rsm}), which combines an out-of-distribution detector (\Cref{txt:ood-detection-chapter}), a prediction model (\Cref{txt:prediction-chapter}), and an uncertainty quantifier (\Cref{txt:uncertainty-chapter}). We also explore how the predicted confidence level and uncertainty range can be propagated through downstream analysis such that they are accurately reflected in the confidence and uncertainty of the final reported result.
    \item We provide a broad literature review on existing out-of-distribution detection methods (\Cref{txt:novelty-detection}), which we then evaluate using a set of toy examples that we introduce (\Cref{txt:ood-detection-comparison} and \Cref{txt:ood-synthesis-analysis}). We also propose $t$-poking (\Cref{fig:fgsm-t-poking-ood-samples}) and in-distribution vs out-of-distribution sample weighting (\Cref{txt:ood-input-generation}) as two add-ons to improve the performance of existing OOD detection methods.
    \item We publish and describe the semi-lagrangian SOSAA dataset (\Cref{txt:sosaa-data-chapter}) that contains the inputs (most notably, the emissions) and outputs of the SOSAA model for 42 runs that follow the air parcel trajectories, calculated with FLEXPART \cite{flexpart-10.4-2019}, arriving at the SMEAR II measurement station in Hyyti\"al\"a, Finland \cite{smear-station-2013} in May 2018. We also include the results of 32 different perturbation runs for six of the trajectories in the dataset. These semi-lagrangian SOSAA datasets are later referred to as SOSAA trajectories.
    \item We evaluate a variety of prediction (\Cref{txt:model-evaluation}) and uncertainty quantification (\Cref{txt:uncertainty-predict-sigma} -- \Cref{txt:uncertainty-ensemble-methods}) methods on the SOSAA trajectories dataset. We also develop a simple clumping technique (\Cref{txt:clumped-train-test-split}) to ensure that the training-test dataset split of time-series data for machine learning does not encourage models to simply interpolate between temporally adjacent values. Additionally, this thesis explores how different uncertainty quantifiers can be evaluated (\Cref{txt:uncertainty-methods}). We then combine an out-of-distribution detector, a prediction model, and an uncertainty quantifier to construct a proof-of-concept SOSAA RSM and stress test it to expose its limitations (\Cref{txt:icarus-evaluation-chapter}).
    \item We also develop and publish a graphical user interface for the SOSAA model to make configuring (\Cref{txt:sosaa-gui-config}), compiling, and running (\Cref{txt:sosaa-gui-compile-run}) the model more accessible to new users. The GUI also includes the functionality to easily train and test our SOSAA RSM implementation (\Cref{txt:sosaa-gui-icarus}).
\end{enumerate}
\noindent Overall, our goal is to study the feasibility of using \textit{prudent} RSMs in research, provide suggestions for how an RSM following the Icarus architecture can be implemented, and identify several areas for future research.

\newpar This thesis is split into three main parts and includes several appendices:

First, an extensive background section reviews theory, methods, and prior work. \Cref{txt:aerosols} briefly summarises aerosols and their impacts, before \Cref{txt:sosaa-model} introduces the \ul{SOSAA model}. Next, \Cref{txt:machine-learning} provides a short introduction to machine learning methods, including Random Forests (see \Cref{txt:ensembles-decision-tree-random-forest}), Pairwise Difference Regression (see \Cref{txt:padre-rf}), and simple neural networks (see \Cref{txt:neural-network}). Following on, \Cref{txt:response-surface-models} gives an overview of the history of \ul{response surface models} in atmospheric science, including milestones such as DeepRSM (see \Cref{txt:deep-rsm}). In \Cref{txt:uncertainty-quantification}, several methods for \ul{quantifying the uncertainty} of prediction models are briefly introduced. Finally, \Cref{txt:novelty-detection} provides an extensive review of \ul{novelty detection methods} for identifying, synthesising, and scoring \ul{out-of-distribution} inputs.

The \ul{Icarus RSM}, which is the core contribution of this thesis, is motivated and presented in \Cref{txt:icarus-chapter}. Next, \Cref{txt:sosaa-data-chapter} introduces and explores the SOSAA trajectories dataset, which is used in the following chapters. \Cref{txt:ood-detection-chapter} explores methods for detecting and synthesising out-of-distribution inputs, which are evaluated using toy examples and the SOSAA dataset. Next, \Cref{txt:prediction-chapter} continues with the second key component of Icarus and analyses the performance of several prediction models on the SOSAA dataset. Following on, \Cref{txt:uncertainty-chapter} explores different methods to quantify the predictive uncertainty of the aforementioned models.

Finally, \Cref{txt:icarus-evaluation-chapter} combines all three components into a proof-of-concept implementation of a \ul{SOSAA RSM} that uses the Icarus architecture, which is then evaluated to expose its limitations. Following on, \Cref{txt:sosaa-gui-chapter} briefly introduces the \ul{SOSAA \textbf{G}raphical \textbf{U}ser \textbf{I}nterface} that allows its users to configure, run and compile the SOSAA model and to train and test the new SOSAA RSM. Last but not least, \Cref{txt:conclusions-chapter} summarises the main \ul{conclusions} of this thesis and highlights some opportunities for \ul{future work}.

\newpar We hope that our framework for building \textit{prudent} response surface models using the Icarus RSM architecture inspires future work into more advanced implementations and enables the safe integration of machine-learning-based RSMs into more research. While these methods hold great promise, they can only be employed with \textit{confidence} and \textit{certainty} if conditioning on confidence scores and uncertainty levels are treated as critical parts of their design. The confidence and uncertainty of such \textit{prudent} predictions can be propagated through analyses and thus allow for the rigorous analysis of any downstream conclusions.

In particular, we hope that the new SOSAA GUI and RSMs built with Icarus support the integration of higher complexity atmospheric chemistry models into policy-making to fight climate change.
